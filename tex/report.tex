\documentclass{llncs}
\usepackage[utf8]{inputenc}
\usepackage{fancyvrb} 
\usepackage[portuguese]{babel}
\usepackage{ragged2e}

\begin{document} \mainmatter
\title{Conversor de ficheiros GPX para KML}
\titlerunning{Conversor de ficheiros GPX para KML}
\author{José Carlos Lima Martins A78821 \and
        Miguel Miranda Quaresma A77049}
\authorrunning{José Carlos Lima Martins A78821 \and
        Miguel Miranda Quaresma A77049}
\institute{                                                                
University of Minho, Department of  Informatics, Braga, Portugal\\
e-mail: \{a78821,a77049\}@alunos.uminho.pt
}

\maketitle

\justify

\begin{abstract}
    Formatos como \textbf{GPX} (GPS Exchange Format) e \textbf{KML} (Keyhole Markup Language) são usados na representação de dados geográficos no entanto estes apresentam propósitos diferentes e, consequentemente, estruturas diferentes. Este relatório documenta o desenvolvimento de um conversor entre ficheiros GPX e ficheiros KML usando, para isso, o analisador léxico \textbf{Flex} (fast lexical analyzer generator). O conversor deve gerar ficheiros (KML) que possam ser inspeccionados com recurso a ferramentas como Google Earth, Maps.me, etc.
\end{abstract}

\section{Estrutura do Relatório}
Após uma breve introdução serão apresentadas as estruturas dos diferentes ficheiros GPX e KML (no Preâmbulo). De seguida é explicado, de forma sucinta, o problema presente para, de seguida, apresentar a solução por nós proposta. Finalmente, e após demoonstrar exemplos de uso e os casos de teste, será apresentada uma breve conclusão em relação aos resultados ao trabalho desenvolvido.

\section{Introdução}
A conversão entre tipos foi feita usando o FLEX, uma ferramenta poderosa que permite criar analisadores léxicos de modo a reconhecer padrões no input e, consoante o padrão, tratar os dados de determinada maneira tendo em conta o resultado pretendido. É importante destacar que o uso  de uma ferramenta como o FLEX simplifica o processo de desenvolvimento visto que a complexidade de implementar algo que desse os mesmos resultados e o tempo de implementação seria consideravelmente superior dado a complexidade de código de deteção de padrões.

\section{Preâmbulo}\label{pb}
De modo a compreender-se melhor o que foi desenvolvido é importante conhecer as estruturas (tags) dos dois tipos de ficheiros. Ambos são baseados em XML (Extensible Markup Language) e como tal apresentam algumas semelhanças, diferindo maioritariamente nas tags e os seus significados. 

Em relação à estrutura do GPX é importante referir que:
\begin{itemize}
    \item \verb|<gpx creator="..." version=".." ..>| raiz do documento XML, indica as versões(\textit{standards}) dos diversos campos presentes no documento
    \item \verb|<metadata>| contém informação sobre o documento GPX: autor do documento, altura de criação, restrições de direito de autor, altura de criação entre outros, etc
    \item \verb|<wpt lat="..." lon="...">| representa um \textit{waypoint} definido  localizado nas coordenadas dadas
        \subitem \verb|<name>| indica o nome do waypoint referido
        \subitem \verb|<time>| indica o timestamp referente à passagem no waypoint
        \subitem \verb|<ele>| indica a altitude do waypoint
        \subitem \verb|<desc>| breve descrição do waypoint
    \item \verb|<trk>| representa um \textit{track} \textbf{i.e.} um conjunto de pontos que compõem um percurso
        \subitem \verb|<name>| indica o nome do percurso
        \subitem \verb|<desc>| descrição do percuro
        \subitem \verb|<type>| indica o tipo de percurso a que se refere(caminhada, ciclismo, etc)
        \subitem \verb|<extensions>| contém informações adicionais como distancia, desnível positivo/negativo, calorias queimadas
    \item \verb|<trkseg>| representa um segmento contínuo de pontos
        \subitem \verb|<trkpt lat="..." lon="...">| ponto constituinte do segmento com as coordenadas indicadas
    \item
        
\end{itemize}

Já em relação à estrutura do KML é de destacar o seguinte:
\begin{itemize}
    \item \verb|<?xml version="1.0" encoding="utf-8"?>|: indica a versão do XML e a encodificação 
    \item \verb|<kml xmlns="http://www.opengis.net/kml/2.2">|: indica a versão do KML usado, pode ser também adicionado extensões dentro da tag
    \item \verb|<Placemark>|: representa como o próprio nome indica um marcador de um local/trilho/ponto/etc
    \item \verb|<name>|: presente tanto no \verb|<Document>| como no \verb|<Placemark>| permite dar um nome ao documento ou ao local respetivamente
    \item \verb|<description>|: como o nome permite descrever/caracterizar o \verb|<Document>| ou o \verb|<Placemark>|
    \item \verb|<LineString>|: permite representar um caminho/trilho, deve estar dentro da tag \verb|<Placemark>|
    \item \verb|<Point>|: permite representar um ponto/marco no mapa, deve estar dentro da tag \verb|<Placemark>|
    \item \verb|<coordinates>|: é aqui adicionado as coordenadas e alturas de cada ponto do caminho a ser ``desenhado'' no caso de estar dentro de uma tag \verb|<LineString>|, caso esteja dentro de um \verb|<Point>| permite indicar a localização e altura do ponto, sendo que a organização é longitude,latitude,altura por cada linha
    \item \verb|<altitudeMode>|: permite defenir que tipo de modo de altitude é usada, decidimos no nosso caso usar absolute (ou seja, a altitude é referente ao nível do mar), deve também estar dentro da tag \verb|<LineString>| ou da tag \verb|<Point>|
\end{itemize}

\section{Análise e especificação}

\subsection{Descrição informal do problema}
O problema consiste na conversão de um ficheiro GPX, gerado por navegadores GPS, para um ficheiro KML, criado e mantido pela gigante Google. Isto envolve conhecer a estrutura de cada ``linguagem''(\ref{pb}). A maior dificuldade reside no facto de que existem tags presentes numa linguagem que não existem na outra e vice-versa, criando um problema de decisão e levando a que seja necessário escolher a tag que melhor representa a informação presente no ficheiro de origem.

\section{Concepção/desenho da Resolução}

\subsection{Estruturas usadas}

\subsection{Algoritmos(Lógica)}
De certa maneira o que é realizado é sempre que algo aparece (condição) realizamos uma ação, neste caso escrever para um ficheiro algo que corresponda com o match da condição. Contudo, devido às diferentes tags entre os dois tipos, esta conversão nem sempre é imediata/trivial e exige o recurso a estratégias que serão explicitadas de seguida.

Uma das estratégias referidas é o uso de condições de contexto de modo a isolar as diferentes secções nas situações em que, por exemplo, existem tags do mesmo tipo mas que são "filhas" de tags diferentes, representando coisas diferentes. Estas condições de contexto permitem ainda o desenvolvimento de um código mais conciso e, em determinados casos, mais simples.

Outra estratégia foi o uso de uma stack que permite manter um registo sobre as condições de contexto anteriores à atual de modo a permitir que as mudanças de contexto ocorram sem ambiguidades derivadas da exitência de múltiplos pontos de entrada para determinadas condições de contexto. Um exemplo disto é a condição de contexto LINK e a sua stack \textit{stackLink}.

Por fim outra estratégia usada, foi o uso de um ``boolean'' na condição de contexto METADATA visto que, como a tag \texttt{name} do GPX deve ficar fora da tag description do KML e que os restantes casos do mesmo(METADATA) devem ficar dentro da tag description este boolean foi usado de modo a sabermos se já tinha sido no ficheiro \verb|<description>| e em caso negativo fazê-lo. Caso já se tenha imprimido \verb|<description>| e apareça a tag name esta é ignorada e não é imprimida de modo a não criar incongruências.

\section{Codificação e Testes}
Com o objetivo de testar a ferramente desenvolvida foram transferidos alguns ficheiros presentes em https://www.openstreetmap.org/traces (2673775.gpx, 2673778.gpx, 2673781.gpx, 2684029.gpx, 2684078.gpx) bem como algumas rotas desenvolvidas com recurso à ferramenta \textbf{Strava} (Small.gpx, Route.gpx).

De seguida apresenta-se o processo recomendado para testar a ferramente desenvolvida:
\begin{itemize}
    \item Compilar o ficheiro gpxToKml.l com o FLEX: \verb|flex gpxToKml.l|
    \item Compilar o ficheiro lex.yy.c: \verb|gcc -o gpxToKml lex.yy.c|
    \item Executar: \verb|./gpxToKml < ficheiroInput.gpx > ficheiroOutput.kml|
    \item Importar para o Google Earth, etc
\end{itemize}

Ou, devido à existência de uma Makefile realizar simplesmente:
\begin{itemize}
    \item \verb|make install|
    \item \verb|./gpxToKml < ficheiroInput.gpx > ficheiroOutput.kml|
    \item Importar para o Google Earth, etc
\end{itemize}

Sendo também possível, com a Makefile, limpar os ficheiros criados usando \verb|make clean|.

\section{Conclusão}
Em conclusão, o resultado obtido foi bom, visto que é possível visualizar o trajeto presente no GPX no Google Earth, etc através da importação do ficheiro KML gerado pelo conversor. Contudo, de modo a melhorar o conversor, poderiamos realizar a conversão das restantes tags do GPX para KML. É importante concluir também que devido ao uso do FLEX com os seus poderosos mecanismos tais como as condições de contexto, permitiu que a resolução do problema fosse mais rápido de obter, mais simples e com bem menos linhas de código, mantendo-se contudo de fácil interpretação.

\appendix
\section{Código do Programa}

\end{document}
