\documentclass{llncs}
\usepackage[utf8]{inputenc}
\usepackage{fancyvrb} 
\usepackage[portuguese]{babel}
\usepackage{ragged2e}

\begin{document} \mainmatter
\title{Conversor de ficheiros GPX para KML}
\titlerunning{Conversor de ficheiros GPX para KML}
\author{José Carlos Lima Martins A78821 \and
        Miguel Miranda Quaresma A77049}
\authorrunning{José Carlos Lima Martins A78821 \and
        Miguel Miranda Quaresma A77049}
\institute{                                                                
University of Minho, Department of  Informatics, Braga, Portugal\\
e-mail: \{a78821,a77049\}@alunos.uminho.pt
}

\maketitle

\justify

\begin{abstract}
De modo a converter um ficheiro GPX (GPS Exchange Format) para KML (Keyhole Markup Language) foi usado o FLEX (fast lexical analyzer generator). Esta conversão permite que a partir de ficheiros gerados pelos gps's por exemplo seja possível ver o conteúdo guardado pelo mesmo no Google Earth (Maps.me, etc).
\end{abstract}

\section{Introdução}

\section{Preliminares}
De modo a compreender-se melhor o que foi desenvolvido é importante conhecer as estruturas (tags) dos dois tipos. Os dois são baseados no XML (Extensible Markup Language) e como tal tem algumas parecências mudando principalmente as tags e os seus significados.

Em relação à estrutura do GPX é importante referir que:
\begin{itemize}
    \item
\end{itemize}

Já em relação à estrutura do KML é de destacar o seguinte:
\begin{itemize}
    \item \verb|<?xml version="1.0" encoding="utf-8"?>|: indica a versão do XML e a condificação 
    \item \verb|<kml xmlns="http://www.opengis.net/kml/2.2">|: indica a versão do KML usado, pode ser também adicionado extensões dentro da tag
    \item \verb|<Placemark>|: representa como o próprio nome indica um marcador de um local/trilho/ponto/etc
    \item \verb|<name>|: presente tanto do \verb|<Document>| como no \verb|<Placemark>| permite dar um nome ao documento ou ao local respetivamente
    \item \verb|<description>|: como o nome permite descrever/caracterizar o \verb|<Document>| ou o \verb|<Placemark>|
    \item \verb|<LineString>|: permite representar um caminho/trilho, deve estar dentro da tag \verb|<Placemark>|
    \item \verb|<Point>|: permite representar um ponto/marco no mapa, deve estar dentro da tag \verb|<Placemark>|
    \item \verb|<coordinates>|: é aqui adicionado as coordenadas e alturas de cada ponto do caminho a ser ``desenhado'' no caso de estar dentro de uma tag \verb|<LineString>|, caso esteja dentro de um \verb|<Point>| permite indicar a localização e altura do ponto, sendo que a organização é longitude,latitude,altura por cada linha
    \item \verb|<altitudeMode>|: permite defenir que tipo de modo de altitude é usada, decidimos no nosso caso usar absolute (ou seja, a altitude é referente ao nível do mar), deve também estar dentro da tag \verb|<LineString>| ou da tag \verb|<Point>|
\end{itemize}

\section{Desenvolvimento}
Explicar o que foi realizado, estruturas usadas e usage

\section{Conclusão}

\end{document}
