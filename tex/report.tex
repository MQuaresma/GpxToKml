\documentclass{llncs}
\usepackage[utf8]{inputenc}
\usepackage{fancyvrb} 
\usepackage[portuguese]{babel}
\usepackage{ragged2e}

\begin{document} \mainmatter
\title{Conversor de ficheiros GPX para KML}
\titlerunning{Conversor de ficheiros GPX para KML}
\author{José Carlos Lima Martins A78821 \and
        Miguel Miranda Quaresma A77049}
\authorrunning{José Carlos Lima Martins A78821 \and
        Miguel Miranda Quaresma A77049}
\institute{                                                                
University of Minho, Department of  Informatics, Braga, Portugal\\
e-mail: \{a78821,a77049\}@alunos.uminho.pt
}

\maketitle

\justify

\begin{abstract}
De modo a converter um ficheiro GPX (GPS Exchange Format) para KML (Keyhole Markup Language) foi usado o FLEX (fast lexical analyzer generator). Esta conversão permite que a partir de ficheiros gerados pelos gps's por exemplo seja possível ver o conteúdo guardado pelo mesmo no Google Earth (Maps.me, etc).
\end{abstract}

\section{Introdução}

\section{Preliminares}
Explicar estruturas de GPX e de KML

\section{Desenvolvimento}
Explicar o que foi realizado, estruturas usadas e usage

\section{Conclusão}

\end{document}
